
% \subsection*{Signaling theory}

% Choosing a hotel sight unseen forces consumers to rely on the quality signals he or she can receive from the hotel. These signals can be advertising, brand name, word of mouth, etc.. Signaling is an important component of our proposed theoretical model. It has a long tradition in economics and marketing. In our study, the first issue to address is the adverse-selection problem. \citet{akerlof1978market} sheds light on the adverse-selection problem, where used cars illustrate the author’s findings. Given a seller of a used car and there is a buyer and there isn’t a third party to inform the buyer of the quality of the car, the seller knows more than the buyer. This information asymmetry creates a “market for lemons” because the value of the car cannot be agreed upon between the buyer and seller. Thus, in equilibrium, only the lowest quality cars will be sold in the market because buyers will be unwilling to pay for higher quality. \citet{spence1978job} offered a solution to this market failure which a partial equilibrium where there is a costly signal firms can provide. The example used is job candidates. There are two students, one is smart and hardworking and the other is the opposite. The first student will have a lower cost of finishing college compare to the second student for with graduating is not possible. Thus only those students with a lower cost will graduate and get hired at better paying jobs. This creates a separating equilibrium. Extending Spence’s separating equilibrium is the seminal work by \citet{milgrom1986price} where they endogenize the cost in their model. For example, two firms are exogenously given product quality levels; high or low. The consumers do not know the quality levels of the products. Firms use prices and advertising as signaling strategies to inform consumers of their product quality. The penalty here comes when a firm lies about the product's quality and charges a high price for a low quality product. The customer will not purchase that product the next time. That potential loss of customers by falsely signaling quality is endogenizing the cost, which is an opportunity cost for future sales. These two signals create a separating equilibrium. This ties in to our analysis between chains and independents. \citet{erdem1998brand} empirically test the brand equity signal between national brands and private labels. \citet{erdem2008dynamic} structural model to show how pricing discount drove the signal up of the brand. However, if the price discounts were frequent then the brand equity decreased.

\textbf{The academic literature studies the interaction of reviews and consumers,  ...cite...} 

The amplification of the consumer's voice via online review platforms have shifted the balance of power over product quality information in the market away from firms' marketing activities towards consumers' knowledge sharing. \citet{chen2008online} argue that online consumer reviews are a form of product information created by users based on personal usage experience and are in effect free “sales assistants” helping consumers identify the products that best match their idiosyncratic usage conditions.
\citet{chen2004impact} show that an increase in information sources could lead to more trust. They show that as the number of consumer reviews increases, the overall rating converges to the true quality. Therefore, reviews of popular products could more accurately reflect product quality and thus could be more influential. A major reason consumers use online reviews is to obtain quality information to reduce risk. \citet{chen2011moderated} find that when being moderated with different probabilities based on their reputations, commentators may display a pattern of reputation oscillation, in which they generate useful content to build up high reputation and then exploit their reputation. Being popular in itself signals higher quality. Previous studies have shown strong linkages between a product’s popularity and its perceived quality. For example, \citet{caminal1996market} develop a model based on market signaling in the presence of imperfect information and find that future consumers interpret popularity or large market shares as a signal of high quality. \citet{hellofs1999market} suggest several mechanisms through which popularity influences perceived quality, such as signaling, creation of network externalities, and inclusion as an attribute in consumers’ quality functions. Despite its managerial importance, research of online reviews on firm quality is scarce. To our knowledge, \citet{smyth2010does} were the first address this question, coining the phrase "TripAdvisor effect". They document the impact of TripAdvisor ratings on hotel service quality and find an increase in service levels as hotels become aware of their image on TripAdvisor. In this paper, we test the hypothesis that democratization of service quality evaluations motivates the improvement of firm quality.
\underline{Papers about online reviews and firms' financial performance:} 

Extant eWOM literature collectively suggests firm performance is increased in a variety of product and service offerings. \citet{zhu2010impact} implement a difference-in-differences approach to document how an increase in online reviews are more influential for less popular video game sales. This finding suggests that the informational role of reviews becomes more salient in an environment in which alternative means of information acquisition are relatively scarce. \citet{seiler2017does} leverage a temporary government banning of the Chinese microblogging platform Sina Weibo due to political events to estimate the causal effect of online word-of-mouth content on product demand in the context of TV show viewership. Given this exogenous variation, they estimate an elasticity of TV show ratings (market share in terms of viewership) with respect to the number of relevant comments (comments were disabled during the government ban) of 0.016. The authors find that more postshow microblogging activity increases demand, whereas comments posted prior to the show airing do not affect viewership. These patterns are inconsistent with informative or persuasive effects and suggest complementarity between TV consumption and anticipated postshow microblogging activity. \citet{ye2009impact} use a log-linear model for online hotel sales to document that positive online reviews can significantly increase the number of bookings in a hotel, and the variance or polarity of eWOM for the reviews of a hotel had a negative impact on the amount of online sales.   

\textbf{, but few studies have examined how online reviews impact firms' behavior....}
\underline{Papers about online reviews and firms' strategic behavior:}

(...transition...) \citet{chen2005third} study how product review information from influential experts affects firms’ pricing and advertising strategies. They show that these reviews can have substitutive and/or complementary effects on firms’ advertising depending on review format and product quality. Advertising to broadcast the good news can in fact hurt products that get favorable expert reviews and it is not necessarily wise to boost advertising expenditures to spread good news.
\citet{ma2015squeaky} study managerial response in a Twitter setting, demonstrating that firms' responses to complaints on Twitter directed towards the firm encourage followers of the complaining customers to voice their own complaints. \citet{wang2018and} find manager responses to negative reviews are seen by subsequent reviewers as valuable service recovery efforts, leading them to rate the hotel more positively. Responses to positive reviews are viewed as promotional activities, leading to more negative subsequent ratings. In addition, they show both effects are magnified when a manager tailors his or her response to its corresponding review. 


\subsection*{Salience theory}

\underline{Papers about psychological grounds of salience (i.e. limited attention):}
Salience, Attention, and Attribution: Top of the Head Phenomena, Shelley E.Taylor Susan T.Fiske
Salience: Attention to and recall of product attributes are determined by their salience, which can be affected by a variety of factors. In the best case, salience will be internally driven and correspond to the importance consumers assign to the attributes. Indeed, there exists a long history of research across several domains showing that recall is related directly to the perceived importance of information (e.g., Johnson 1970; Kintsch and van Dijk 1978; Lichtenstein and Brewer 1980; Voss, Vesonder, and Splich 1980). In some cases, as when particular attributes are personally very important, then will be recalled very well even when processing conditions are adverse (Bargh and Thein 1985). In the extreme case, processing of such information may verge on being automatic. Thus, even under highly distracting conditions, consumers may encode and later recall attributes they consider important for evaluation (see Ratneshwar, Mick, and Reitinger 1989).

Such outcomes are heartening and suggest that consumers will make decisions that, if not objectively optimal, are at least consistent with their personal values. The problem with this scenario, of course, is that consumers do not always posses well formed beliefs about attribute importance (see Fischhoff, Slovic, and Lichtenstein 1980; Jacoby, Troutman, Kuss, and Mazursky 1986; Wind and DeVita 1976). In such cases salience may be determined externally, perhaps via repetition or by the format in which it is conveyed. (see Finn 1988). In addition, there may be characteristics of the information itself that makes it inherently salient. In the context of cuing on brand recall, it has been shown to affect the size and composition of the evoked set. In the context of attributes, such increases in salience may be discussed in terms of "problem framing," whereby the effective weight, of attributes in decision making are altered by manipulations of either the perceptual salience or the memory accessibility of that information (e.g., Gardener 1983; MacKenzie 1986; Wright and Rip 1980).

Wright and Rip (1980) demonstrated that repeated reference to certain product dimensions increased their influence in subsequent judgments by consumers. Other studies have shown that decisions can be influenced more subtly simply by increasing the prominence of some facts without explicitly suggesting that they are very important. 
...

\underline{Papers demonstrating firms' limited attention}

\underline{Papers about behavior near goal thresholds for individuals, firms, and organizations:}

Individuals:
% An individual may be goal-directed (Lee and Ariely 2006) and begin the path process with specific goal(s) in mind (e.g., to buy groceries for dinner, to find a textbook on Amazon.com), or at the other extreme, his trip can be a purely hedonic “browsing” experience. Agents with varying degrees of “goal-directedness” may exhibit very different patterns. For instance, in her study of online shoppers, Moe (2003) identified four types of internet browsers with different goals, “directed buying,” “search and deliberation,” “knowledge building,” and “hedonic browsing,” and found that distinct types of shoppers exhibit very different page-to-page browsing patterns. Web users exhibit different degrees of goal-directed behavior.
\citet{uetake2017success} demonstrate that short-term goal accomplishment may indeed have a positive spillover onto long-term goal accomplishment. A novel large-scale data set from a popular mobile weight management application is used to track the daily dynamics of weight loss and calories consumption across a large number of users. By comparing a various cases of daily budgets for calories in which the user is slightly under or over-budget, the authors provide an empirical link between short-term goal achievement and various long-term outcomes. Using a synthetic control method to predict counterfactual outcomes, they show that the results are robust to potential manipulation of calories consumption around the goal. The estimates from a dynamic structural model of calories management reveal that users receive positive utility from past short-term goal accomplishments, and counterfactual analysis with the estimated model quantify the long-run user benefits of various hypothetical policies that adjust the budget.

Firms and Organizations: 
Analyzing equity pension fund manager investments strategies \citet{lakonishok1991window} find that the managers would "window dress" their portfolios by accelerating the dumping of poorly performing stocks. The authors document an increase in sales of poorly performing stocks in the fourth quarter.
\citet{chevalier1997risk} examine portfolio holdings of mutual funds in September and December and show that mutual funds do alter the riskiness of their portfolios at the end of the year.

% Extant literature points to the pivotal role of goals in our daily lives: they provide us with a sense of direction and clarity for our actions and influence the way that we think and behave (Gollwitzer 1990 “Action Phases and Mindsets,” in Handbook of Motivation and Cognition: Foundations of Social Behavior; Gollwitzer, P. M. (2012). Mindset theory of action phases. In P. Van Lange, A. W. Kruglanski, and E. T. Higgins (Eds.), Handbook of theories of social psychology (pp. 526–545). London: Sage.; Locke and Latham 1990 A Theory of Goal Setting and Task Performance: show higher goal commitment allows for attaining a complex/high leveled goal). For example, when considering managers’ goals, it is difficult to imagine that such goals would be clearly defined at all times. For example, a hotel manager may have a general goal of managing a hotel's reputation, but this goal might later translate into the more specific goal of specifically improving the hotel's room and/or service quality.

% Gollwitzer's (1990, 1999 Implementation Intentions: Strong Effects of Simple Plans) mind-set theory, states that an individuals execution of volitional control involves two phases. In the first phase, individuals are uncertain about their goals; they are in a deliberative mind-set and seek to define “a desired performance or an outcome” (Gollwitzer 1999, 494). In the second phase, individuals have already established their goals, and they switch to an implemental mind-set where they pursue implementation intentions and well-defined “when, where, and how responses leading to goal attainment” (494). Once managers have constructed concrete service quality improvement goals, they move to a second stage, one that is characterized by goal determinism and action tenacity.


\subsection*{Causal inference papers}

\underline{Methods papers about event studies with causal inference:}

Advertising response studies are notoriously plagued by endogeneity. To estimate causal effects, \citet{liaukonyte2015television} investigate whether and how television advertising impact online shopping. The authors build a novel data set with 20 brands, measures of brands’ website traffic and transactions, and ad content measures for 1,224 commercials.  A quasi-experimental design is implemented to estimate whether and how TV advertising influences changes in online shopping within two-minute pre-post windows of time. Using nonadvertising competitors’ online shopping in a difference-in-differences approach the authors measure the same effects in two-hour windows around the time of the ad. They find action-focus content increases direct website traffic and sales, while information-focus and emotion-focus ad content actually reduce website traffic and simultaneously increasing purchases, with a positive net effect on sales for most brands. 
\citet{hollenbeck2017advertising} document how online reviews have had the effect of displacing advertising. Using a novel data set of TripAdvisor hotel reviews and another describing hotels' advertising expenditures, the authors, first, that overall ad spending decreased from 2002 to 2015. Second, there is a negative causal relationship between TripAdvisor ratings and advertising spending in the cross-section: hotels with higher ratings spend less. This suggests that user ratings and advertising are substitutes, not complements. Third, this relationship is stronger for independent hotels than for chains, and stronger in competitive markets than in noncompetitive markets. The former suggests that a strong brand name provides some immunity to reviews, and the latter suggests that when ratings are pivotal, the advertising response might be particularly strong. Finally, they show that the relationship between user ratings and advertising has strengthened over time, as websites such as TripAdvisor have become more influential.

\citet{carpenter2011minimum} who examine the impact of the minimum legal drinking age on mortality. The authors use a panel fixed effects approach and a regression discontinuity approach to estimate the effects of the minimum legal drinking age on mortality, and they also discuss what is known about the relationship between the minimum legal drinking age and other adverse outcomes such as nonfatal injury and crime. They document the effect of the minimum legal drinking age on alcohol consumption and estimate the costs of adverse alcohol-related events on a per-drink basis. \citet{mccarthy2015quality} identify the effect of star ratings on insurance premiums using a regression discontinuity design that exploits plausibly random variation around rating thresholds.  \citet{luca2011reviews}, exploits Yelp rounding of aggregated ratings to the nearest half star in a regression discontinuity design framework to establish the effect of Yelp ratings on restaurant revenue and competition dynamics. 

% Busse, Silva-Risso, and Zettelmeyer
% (2006), \citet{hartmann2011identifying} RDD The shorter the temporal window on each side of the treatment, the less likely that factors besides the treatment will affect outcomes  and Narayanan and Kalyanam (2015).

\underline{Methods papers about time series methods and causal inference (Hal Varian's paper): }

DLM identify structural breaks

Marketing data are often observational and rarely follow the ideal of a randomized design. They usually exhibit a low signal-to-noise ratio with multiple seasonal variations, confounded by the effects of unobserved variables and their interactions. The standard approach to causal inference in such settings is based on a linear model of the observed outcomes in the treatment and control group before and after the intervention. Then the differences between the pre-post difference in the treatment group and the pre-post difference in the control group are estimated. The assumption underlying such difference-in-differences designs is that the level of the control group provides an adequate proxy for the level that would have been observed in the treatment group in the absence of treatment.  \citet{brodersen2015inferring} propose a method to infer the causal impact on the basis of a diffusion-regression state-space model that predicts the counterfactual market response in a synthetic control that would have occurred had no intervention taken place. The state-space model allows to infer the temporal evolution of attributable impact, incorporate empirical priors on the Bayesian parameters, include local trends, seasonality and the time-varying influence of contemporaneous covariates. The authors describe a case where the outcome of interest was visits to a particular website and the treatment was advertising spend. They show that the number of “searches” about topics related to the subject matter of the website predict the number of visits to this website..

\citet{abadie2010synthetic} examine the application of synthetic control methods to comparative case studies. The authors document the advantages of these methods and apply them to study the effects of Proposition 99, a large-scale tobacco control program that California implemented in 1988. The authors demonstrate that, following Proposition 99, tobacco consumption fell markedly in California relative to a comparable synthetic control region. They estimate that by the year 2000 annual per-capita cigarette sales in California were about 26 packs lower than what they would have been in the absence of Proposition 99.

\underline{Methods papers about causal inference in online reviews context:}

\citet{chevalier2006effect} use a difference-in-differences approach to establish the causal relationship between negative reviews and relative book sales on Amazon.com and BN.com. \citet{xu2017doctordemand} examine social media platforms for healthcare services and derive various service-quality proxies from online reviews and study the relationship between these quality proxies and physician demand. The authors examine a unique data set from one of the leading appointment booking websites in the United States, that contains online physicians' appointments made over a five-month period, along with other online information. They propose a random coefficient choice model to characterize patient heterogeneity in physician choices, taking into account both numeric and textual user-generated content with text mining techniques.

\citet{proserpio2017online} use a difference-in-differences identification strategy to analyze a sample of mixed-level hotels to argue that managerial response to online reviews discourages negative reviewers from reviewing, thus provide a higher subsequent hotel rating. \citet{wang2018and} estimated the average treatment and treatment on the treated effects of manager responses on subsequent review ratings using hotel reviews from various review platforms. \citet{luca2011reviews} exploits Yelp rounding of aggregated ratings to the nearest half star in a regression discontinuity design framework to establish the effect of Yelp ratings on restaurant revenue and competition dynamics. To test whether online reviews have a causal effect on demand, \citet{ghose2012designing} implement a regression discontinuity design to identify any discontinuous jumps in sales patterns following discontinuous jumps in the rounded overall hotel rating. The authors find a significant positive treatment effect, suggesting that the discontinuous pattern in the sales is caused by the discontinuous pattern in the rating. This finding strongly suggests online reviews have a causal impact on hotel demand.  The closest research related to our study, is a working paper by \citet{hollenbeck2017online} where he investigates the value of branding and how it is changing in response to a large increase in consumer information provided by online reputation mechanisms. Using a 15 year panel of hotel revenues with millions of online reviews from multiple platforms, the author finds that hotel brands earn substantially higher revenues than equivalent independent hotels, but that this premium has declined by over 50 percent from 2000 to 2015. This can be largely attributed to an increase in online reputation mechanisms, and that this affect is largest for low quality and small market firms. Our study ...

